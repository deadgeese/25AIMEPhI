\documentclass[a4paper,12pt]{article}
	\usepackage[left=2cm,right=1cm,top=1cm,bottom=1.5cm]{geometry}
	\usepackage[utf8]{inputenc}
	\usepackage[english,russian]{babel}
	\usepackage{graphicx}
	\usepackage{amsmath}
	\usepackage{amssymb}
	\usepackage{cite}
	\usepackage{indentfirst}
	\usepackage{multicol}
    \usepackage{booktabs}
	\usepackage{cmap}
    \usepackage{multirow}
    \usepackage{longtable}
    \usepackage{geometry}
    \usepackage{float}
    \usepackage{caption}
    \usepackage{subcaption}
    \usepackage{siunitx}
	
	\sloppy
	
	\usepackage{geometry}
	\geometry{top=2cm}
	\geometry{bottom=2cm}
	\geometry{left=2.5cm}
	\geometry{right=2.5cm}
	
	\renewcommand{\baselinestretch}{1.5}
	
	\begin{document}
		\renewcommand{\contentsname}{\Large Содержание}
		\renewcommand{\bibname}{\normalfont\Large\bfseries Список литературы}
		
		\begin{titlepage}
			\begin{center}
				Министерство науки и высшего образования Российской Федерации \\
				НАЦИОНАЛЬНЫЙ ИССЛЕДОВАТЕЛЬСКИЙ ЯДЕРНЫЙ УНИВЕРСИТЕТ <<МИФИ>> \\*
				\hrulefill\\
			\end{center}
		
		\begin{center}
			ИНСТИТУТ ЛАЗЕРНЫХ И ПЛАЗМЕННЫХ ТЕХНОЛОГИЙ\\
			КАФЕДРА №31 ПРИКЛАДНАЯ МАТЕМАТИКА
		\end{center}
		\vspace{1cm}
		
		\vspace{2em}
		
		\begin{center}
			\large{Отчет}
		\end{center}
		
		\begin{center}
			\large По анализу закономерностей глобальных климатических событий, 
            их экономических последствий и влияния на население и инфраструктуру.
		\end{center}
		
		\begin{center}
			Автор: \\
			Соколов П.В.
		\end{center}
	
\vspace{28em}
		
		\begin{center}
			г. Москва 2025
		\end{center}
	\end{titlepage}

\newpage
\tableofcontents

\newpage

\setcounter{page}{2}
\section{Введение}
\subsection{Цель исследования}
Анализ закономерностей глобальных климатических событий, 
их экономических последствий и влияния на население и нфраструктуру для 
выявления ключевых факторов риска ущерба.

\subsection{Методология}
Исследование основано на статистическом анализе данных, включая 
описательную статистику, корреляционный анализ, проверку 
статистических гипотез с использованием критериев Пирсона, Спирмена и Крускала-Уоллиса.

\section{Методы и данные}

\subsection{Описание набора данных}
Набор данных содержит информацию о 3000+ климатических событиях за 2020-2025 годы со следующими характеристиками:

\begin{itemize}
\item \textbf{Временные параметры}: дата, год, месяц
\item \textbf{Географические параметры}: страна, координаты
\item \textbf{Характеристики событий}: тип, тяжесть, продолжительность
\item \textbf{Последствия}: затронутое население, жертвы, экономический ущерб
\item \textbf{Инфраструктура}: ущерб инфраструктуре, время реагирования
\end{itemize}

\subsection{Статистические методы}
\begin{itemize}
\item Описательная статистика
\item Корреляционный анализ Пирсона
\item Корреляционный анализ Спирмена
\item Однофакторный дисперсионный анализ (тест Крускала-Уоллиса)
\item Визуализация распределений
\end{itemize}

\section{Результаты}

\subsection{Описательная статистика}

\begin{table}[h]
\centering
\caption{Основные характеристики набора данных}
\begin{tabular}{lrr}
\toprule
Параметр & Значение \\
\midrule
Общее количество событий & 3000 \\
Количество стран & 51 \\
Период наблюдения & 2020-2025 гг. \\
Средняя тяжесть события & 4.2 \\
Максимальный экономический ущерб & 718.21 млн \$ \\
Общий экономический ущерб & 5833.8 млн \$ \\
\bottomrule
\end{tabular}
\end{table}

\subsection{Распределение событий по типам}

\begin{figure}[h!]
		\begin{center}
			\includegraphics[scale=0.7]{images/clim_ev.png}
		\end{center}
		\caption{Распределение климатических событий по типам}\label{fig:fig1}
\end{figure}

\newpage

\subsection{Экономический ущерб по типам событий}

\begin{figure}[h!]
		\begin{center}
			\includegraphics[scale=0.9]{images/eco_imp.png}
		\end{center}
		\caption{Суммарный экономический ущерб по типам событий}\label{fig:fig2}
\end{figure}
\newpage

\subsection{Матрицы корреляций}

\begin{figure}[h!]
		\begin{center}
			\includegraphics[scale=0.6]{images/corr_matr.png}
		\end{center}
		\caption{Матрица корреляций Пирсона}\label{fig:fig3}
\end{figure}

\begin{figure}[h!]
		\begin{center}
			\includegraphics[scale=0.6]{images/spear_corr.png}
		\end{center}
		\caption{Матрица корреляций Спирмена для рассматриваемых значений}\label{fig:fig4}
\end{figure}

\subsection{Географический анализ}

\subsubsection{Распределение по климатическим зонам}

\begin{table}[H]
\centering
\caption{Распределение событий по климатическим зонам}
\begin{tabular}{lrrrrr}
\toprule
Климатическая зона & Количество & Средний ущерб & Суммарный ущерб & Средние жертвы \\
\midrule
Тропическая & 412 & 2.34 & 964.2 & 45.2 \\
Субтропическая & 387 & 3.12 & 1207.4 & 38.7 \\
Умеренная & 521 & 2.89 & 1505.9 & 52.1 \\
Субполярная & 198 & 1.87 & 370.3 & 28.9 \\
Полярная & 50 & 0.95 & 47.5 & 12.3 \\
\bottomrule
\end{tabular}
\end{table}

\subsubsection{Сравнение полушарий}

\begin{table}[H]
\centering
\caption{Сравнение характеристик событий по полушариям}
\begin{tabular}{lrrr}
\toprule
Параметр & Северное & Южное & Отношение \\
\midrule
Количество событий & 1478 & 1522 & 1:1.03 \\
Средний экономический ущерб & 1.65 & 2.23 & 1:1.35 \\
Суммарный экономический ущерб & 2441.3 & 3391.8 & 1:1.38 \\
Средние жертвы & 44.58 & 43.13 & 1.03:1 \\
Средняя тяжесть & 3.84 & 3.74 & 1.05:1 \\
\bottomrule
\end{tabular}
\end{table}


\subsection{Анализ факторов человеческих жертв}

\subsubsection{Сравнение типов событий}

\begin{figure}[h!]
		\begin{center}
			\includegraphics[scale=0.6]{images/mean_cas.png}
		\end{center}
		\caption{Среднее количество жертв по типам событий}\label{fig:fig5}
\end{figure}

\begin{figure}[h!]
		\begin{center}
			\includegraphics[scale=0.6]{images/sum_cas.png}
		\end{center}
		\caption{Общее количество жертв по типам событий}\label{fig:fig6}
\end{figure}

Статистический тест Крускала-Уоллиса показал значимые различия между типами событий (H = 45.672, p < 0.001).

\subsubsection{Корреляционный анализ}

\begin{table}[H]
\centering
\caption{Корреляции Спирмена с количеством жертв}
\begin{tabular}{lr}
\toprule
Фактор & Коэффициент корреляции \\
\midrule
Ущерб инфраструктуре & 0.672*** \\
Длительность события & 0.451** \\
Тяжесть события & 0.389** \\
Пораженное население & 0.324* \\
Экономический ущерб & 0.312* \\
Время реагирования & 0.187 \\
\bottomrule
\end{tabular}
\end{table}

\begin{itemize}
	\item ***~-- $p < 0.001$
	\item **~-- $p < 0.01$
	\item *~-- $p < 0.05$
\end{itemize}


\section{Проверка статистических гипотез}

\subsection{Гипотеза 1: Различия в средней тяжести между типами событий}

\textbf{Нулевая гипотеза}: Средняя тяжесть не различается между типами климатических событий.

\textbf{Результаты теста Крускала-Уоллиса}:
\begin{itemize}
\item H-статистика: 718.542
\item p-value: 0.0001
\end{itemize}

\textbf{Вывод}: Существуют статистически значимые различия в медианной тяжести между типами климатических событий ($p < 0.001$).

\subsection{Гипотеза 2: Корреляция между тяжестью и экономическим ущербом}

\textbf{Нулевая гипотеза}: Нет корреляции между тяжестью события и экономическим ущербом.

\textbf{Результаты корреляционного анализа}:
\begin{itemize}
\item Коэффициент корреляции Пирсона: 0.119
\item p-value: 0.0014
\item Коэффициент корреляции Спирмена: 0.567
\item p-value: 0.001
\end{itemize}

\textbf{Вывод}: Отвергаем нулевую гипотезу. Обнаружена статистически значимая положительная корреляция между тяжестью события и экономическим ущербом ($r = 0.12$, $p < 0.01$).

\subsection{Гипотеза 3: Корреляция между жертвами и ущербом инфраструктуре}

\textbf{Нулевая гипотеза}: Нет корреляции между жертвами события и ущербом инфраструктуре.

\textbf{Результаты корреляционного анализа}:
\begin{itemize}
\item Коэффициент корреляции Пирсона: 0.44
\item p-value: 0.0031
\item Коэффициент корреляции Спирмена: 0.658
\item p-value: 0.0001
\end{itemize}

\textbf{Вывод}: Отвергаем нулевую гипотезу. Обнаружена статистически значимая положительная корреляция между тяжестью события и экономическим ущербом ($r = 0.44$, $p < 0.01$).

\subsection{Гипотеза 4: Корреляция между жертвами и экономическим ущербом}

\textbf{Результаты корреляционного анализа}:
\begin{itemize}
\item Коэффициент корреляции Пирсона: 0.103
\item p-value: 0.00015
\item Коэффициент корреляции Спирмена: 0.547
\item p-value: 0.00012
\end{itemize}

\textbf{Вывод}: Обнаружена слабая, но статистически значимая положительная корреляция между количеством жертв и экономическим ущербом ($r = 0.312$, $p < 0.05$).

\subsection{Гипотеза 5: Различия в экономическом ущербе между странами}

\textbf{Результаты теста Крускала-Уоллиса для топ-10 стран}:
\begin{itemize}
\item H-статистика: 121.6
\item p-value: 0.0001
\end{itemize}

\textbf{Вывод}: Существуют статистически значимые различия в медианном экономическом между различными странами ($p < 0.001$).

\subsection{Гипотеза 6: Влияние времени реагирования на количество жертв}

\textbf{Результаты корреляционного анализа}:
\begin{itemize}
\item Коэффициент корреляции Пирсона: 0.026
\item p-value: 0.1541
\item Коэффициент корреляции Спирмена: 0.005
\item p-value: 0.7669
\end{itemize}

\textbf{Вывод}: Не обнаружено статистически значимой корреляции между временем реагирования и количеством жертв ($p > 0.05$).

\section{Вывод}

\subsection{Основные закономерности}

\begin{enumerate}
\item \textbf{Типы событий}: Ураганы и засухи являются наиболее частыми типами климатических событий, в то время как тепловые волны наносят наибольший экономический ущерб.

\item \textbf{Экономические последствия}: Общий экономический ущерб превысил 5.8 миллиардов долларов, причем тепловые волны, наводнения и ураганы составляют более 60\% от общей суммы.

\item \textbf{Статистические зависимости}: Выявлены значимые корреляции между тяжестью событий и экономическим ущербом, а также между количеством жертв и экономическими потерями.

\item \textbf{Географические различия}: Наблюдаются значительные различия в экономическом ущербе между странами, что указывает на различную уязвимость к климатическим катастрофам.
\end{enumerate}

\section{Заключение}

Проведенный анализ демонстрирует сложную структуру взаимосвязей между различными параметрами климатических событий. Выявленные закономерности позволяют лучше понять факторы, влияющие на масштабы последствий климатических катастроф, и разработать более эффективные стратегии управления рисками. Статистически значимые корреляции подтверждают важность комплексного подхода к анализу климатических рисков, учитывающего как природные, так и социально-экономические факторы.

Дальнейшие исследования должны быть направлены на анализ более длительных временных рядов, изучение влияния климатических изменений на частоту и интенсивность событий, а также разработку прогнозных моделей для оценки потенциальных рисков.

\end{document}


% \subsection{Кластерный анализ географических данных}

% K-means кластеризация выявила 4 географических кластера с различными характеристиками:

% \begin{table}[H]
% \centering
% \caption{Характеристики географических кластеров}
% \begin{tabular}{lrrrrr}
% \toprule
% Кластер & Средняя широта & Средний ущерб & Средние жертвы & Преобладающий тип \\
% \midrule
% 1 & 45.2°N & 3.24 & 56.3 & Наводнение \\
% 2 & 15.3°N & 2.87 & 42.1 & Ураган \\
% 3 & 35.6°S & 2.15 & 38.7 & Засуха \\
% 4 & 28.4°N & 4.12 & 61.8 & Тепловая волна \\
% \bottomrule
% \end{tabular}
% \end{table}
