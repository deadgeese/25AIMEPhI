\documentclass[a4paper,12pt]{article}
	\usepackage[left=2cm,right=1cm,top=1cm,bottom=1.5cm]{geometry}
	\usepackage[utf8]{inputenc}
	\usepackage[english,russian]{babel}
	\usepackage{graphicx}
	\usepackage{amsmath}
	\usepackage{amssymb}
	\usepackage{cite}
	\usepackage{indentfirst}
	\usepackage{multicol}
	\usepackage{cmap}
	
	\sloppy
	
	\usepackage{geometry}
	\geometry{top=2cm}
	\geometry{bottom=2cm}
	\geometry{left=2.5cm}
	\geometry{right=2.5cm}
	
	\renewcommand{\baselinestretch}{1.5}
	
	\begin{document}
		\renewcommand{\contentsname}{\Large Содержание}
		\renewcommand{\bibname}{\normalfont\Large\bfseries Список литературы}
		
		\begin{titlepage}
			\begin{center}
				Министерство науки и высшего образования Российской Федерации \\
				НАЦИОНАЛЬНЫЙ ИССЛЕДОВАТЕЛЬСКИЙ ЯДЕРНЫЙ УНИВЕРСИТЕТ <<МИФИ>> \\*
				\hrulefill
			\end{center}
		
		\begin{center}
			ИНСТИТУТ ЛАЗЕРНЫХ И ПЛАЗМЕННЫХ ТЕХНОЛОГИЙ\\
			КАФЕДРА №31 ПРИКЛАДНАЯ МАТЕМАТИКА
		\end{center}
		\vspace{1cm}
		
		\vspace{2em}
		
		\begin{center}
			\large{ОТЧЕТ}
			
			по работе 
			за осенний семестр 2025 года \\
			
			на тему:
		\end{center}
		
		\begin{center}
			\large Исследование датасета Shinkansen Stations in Japan
		\end{center}
	
	

\vspace{31em}
		
		\begin{center}
			г. Москва 2024
		\end{center}
	\end{titlepage}

	\newpage 
	\tableofcontents
	\setcounter{page}{3}
	
	\newpage
	
	
	
% === Автоматически сгенерированный раздел: Исследование станций Синкансена ===

\section{Введение}
В ходе данного исследования рассматривается развитие и географическое распределение станций скоростной железнодорожной сети Синкансен в Японии. 
Целью работы является выявление закономерностей в распределении станций по времени открытия, префектурам и линиям, а также анализ пространственных характеристик сети.
Актуальность темы обусловлена значительным вкладом системы Синкансен в развитие транспортной инфраструктуры Японии и её влиянием на социально-экономическое развитие регионов.

\section{Описание данных}
Данные для анализа были получены с платформы \textbf{Kaggle}, из открытого датасета \textit{Shinkansen Stations in Japan}, содержащего сведения о станциях высокоскоростной железной дороги Японии. 
В наборе данных представлены названия станций, годы открытия, принадлежность к префектурам и линиям, а также координаты, позволяющие проводить пространственный анализ.
Данные были предварительно очищены и структурированы с использованием инструментов библиотеки \texttt{pandas}.

\section{Методы исследования}
В исследовании применялись методы анализа данных с использованием языка программирования \textbf{Python}. 
Для визуализации результатов использовались библиотеки \texttt{matplotlib} и \texttt{geopandas}. 
Были построены графики, отражающие динамику открытия станций, распределение по регионам и линиям, а также рассчитаны средние расстояния между станциями.

\section{Результаты анализа}
В данном разделе представлены основные результаты исследования и соответствующие визуализации.

\subsection{Общее количество станций}
Всего в датасете содержится более сотни станций, относящихся к различным линиям сети Синкансен. 
На рисунке представлена общая характеристика выборки.

\subsection{Динамика открытия станций по годам}
На следующей визуализации (рисунок~\ref{fig:figure_2}) представлена динамика открытия станций по годам. 
Анализ показал, что пик ввода в эксплуатацию новых станций пришёлся на периоды активного расширения сети, в особенности на 1970–1990-е годы. 
Отдельно выделен Токио, являющийся центральным транспортным узлом страны.

\begin{figure}[ht]
\centering
\includegraphics[width=0.8\textwidth]{figure_1.png}
\caption{Количество открытых станций по годам с выделением Токио и островов.}
\label{fig:figure_2}
\end{figure}

\subsection{Распределение станций по префектурам}
На рисунке~\ref{fig:figure_3} показано распределение станций по префектурам. 
Наибольшая концентрация наблюдается в густонаселённых регионах, включая префектуры Yamagata, Iwate, Niigata. 
Этот результат отражает закономерность территориального развития транспортной инфраструктуры в зависимости от плотности населения и экономической активности.

\begin{figure}[ht]
\centering
\includegraphics[width=0.8\textwidth]{figure_3.png}
\caption{Распределение станций по префектурам Японии.}
\label{fig:figure_3}
\end{figure}

\subsection{Распределение станций по линиям Синкансена}
На рисунке~\ref{fig:figure_4} представлено распределение станций по основным линиям Синкансена. 
Анализ показал, что наиболее протяжённой и развитой является линия \textit{Tōkaidō Shinkansen}, соединяющая Токио, Нагою и Осаку, на долю которой приходится значительная часть станций.

\begin{figure}[h!]
\centering
\includegraphics[width=0.8\textwidth]{figure_4.png}
\caption{Распределение станций по линиям Синкансена.}
\label{fig:figure_4}
\end{figure}

\subsection{Среднее расстояние между станциями}
На рисунке~\ref{fig:figure_5} представлены данные о среднем расстоянии между станциями для каждой линии. 
Результаты демонстрируют, что линии, проходящие через центральные регионы, характеризуются меньшими интервалами между станциями, тогда как северные и западные ветви имеют более протяжённые участки без остановок.

\begin{figure}[ht]
\centering
\includegraphics[width=0.8\textwidth]{figure_5.png}
\caption{Среднее расстояние между станциями на различных линиях.}
\label{fig:figure_5}
\end{figure}

\subsection{Гипотезы}
В ходе исследования были выдвинуты гипотезы о том, что во-первых с увеличением расстояния до Токио уменьшается плотность станций на линиях, а во-вторых плотность станций на линиях зависит от острова, на котором станции расположены. Однако в ходе исследования обе гипотезы были отвергнуты в силу высокой вероятности выполнения обратной гипотезы.
\newpage

\section{Заключение}
В ходе анализа данных о станциях сети Синкансен были выявлены пространственные и временные закономерности развития скоростной железнодорожной инфраструктуры Японии. 
Основные результаты исследования включают:
\begin{itemize}
\item постепенное расширение сети с концентрацией новых станций в периоды экономического роста;
\item выраженную неравномерность распределения станций по префектурам;
\item доминирование линии Tōkaidō Shinkansen как ключевого маршрута;
\item зависимость расстояний между станциями от географических условий и плотности населения.
\end{itemize}
Результаты подтверждают устойчивую тенденцию к сбалансированному развитию транспортной сети с акцентом на основные экономические центры страны.

% === Конец автоматически сгенерированного раздела ===

\end{document}