\documentclass[a4paper,12pt]{article}
	\usepackage[left=2cm,right=1cm,top=1cm,bottom=1.5cm]{geometry}
	\usepackage[utf8]{inputenc}
	\usepackage[english,russian]{babel}
	\usepackage{graphicx}
	\usepackage{amsmath}
	\usepackage{amssymb}
	\usepackage{cite}
	\usepackage{indentfirst}
	\usepackage{multicol}
	\usepackage{cmap}
	\usepackage{hyperref}
	\usepackage{multirow}
	
	\sloppy
	
	\usepackage{geometry}
	\geometry{top=2cm}
	\geometry{bottom=2cm}
	\geometry{left=2.5cm}
	\geometry{right=2.5cm}
	
	\renewcommand{\baselinestretch}{1.5}
	
	\begin{document}
		\renewcommand{\contentsname}{\Large Содержание}
		\renewcommand{\bibname}{\normalfont\Large\bfseries Список литературы}
		
		\begin{titlepage}
			\begin{center}
				Министерство науки и высшего образования Российской Федерации \\
				НАЦИОНАЛЬНЫЙ ИССЛЕДОВАТЕЛЬСКИЙ ЯДЕРНЫЙ УНИВЕРСИТЕТ <<МИФИ>> \\*
				\hrulefill
			\end{center}
		
		\begin{center}
			ИНСТИТУТ ЛАЗЕРНЫХ И ПЛАЗМЕННЫХ ТЕХНОЛОГИЙ\\
			КАФЕДРА №31 ПРИКЛАДНАЯ МАТЕМАТИКА
		\end{center}
		\vspace{1cm}
		
		\vspace{2em}
		
		\begin{center}
			\large Домашнее задание
            по курсу <<Машинное обучение>>
		\end{center}


		\begin{center}
			\large Выборочный анализ данных об авиационных происшествиях
		\end{center}

		\vspace{20em}

		\vbox{%
			\hfill%
			\vbox{%
				\hbox{\large{Отчёт выполнил:}}
				\hbox{Будников Никита Евгеньевич}%
			}%
		}	 
	

\vspace{11em}
		
		\begin{center}
			г. Москва 2025
		\end{center}
	\end{titlepage}

	\newpage
	
	\tableofcontents
	
	\newpage

\section{Введение}
Данный отчёт представляет выборочный анализ набора данных об авиационных
происшествиях за период с $1919$ по $2025$ год. Исследование проводилось с
использованием \verb|Python| в \verb|Jupyter Notebook| и включало анализ $37,557$
записей о авиационных инцидентах.

\subsection{Ссылка на датасет}

\href{https://www.kaggle.com/datasets/ravindrasinghrana/airplane-crashes-since-1919}
{https://www.kaggle.com/datasets/ravindrasinghrana/airplane-crashes-since-1919}

\subsection{Структура данных}
Набор данных содержит 27 признаков, которые можно разделить на несколько категорий.
Наиболее интересной представляется информация, связанная с датой происшествия,
типом воздушного судна, количеством жертв на борту и налётом экипажа данного рейса.
Для более удобной работы с датами был использован встроенный формат данных
\verb|pandas.Timestamp|, позволяющий совершать над датами арифметические операции.

\subsubsection{Основная информация о происшествии}
\begin{itemize}
    \item \textbf{Date} --- дата и время происшествия (парсинг в формате \verb|dd-mm-yyyy HH:MM|)
    \item \textbf{Type\_of\_aircraft} --- тип воздушного судна
    \item \textbf{Operator} --- оператор/авиакомпания
    \item \textbf{Registration} --- регистрационный номер ВС
    \item \textbf{Flight\_Phase} --- фаза полёта во время происшествия
    \item \textbf{Flight\_Type} --- тип полёта (пассажирский, грузовой, учебный и т.д.)
    \item \textbf{Survivors} --- наличие выживших
    \item \textbf{Site} --- место происшествия
    \item \textbf{Schedule} --- расписание/маршрут полёта
\end{itemize}

\subsubsection{Информация о жертвах}
\begin{itemize}
    \item \textbf{Crew\_on\_board} --- количество членов экипажа на борту
    \item \textbf{Crew\_fatalities} --- количество погибших членов экипажа
    \item \textbf{Pax\_on\_board} --- количество пассажиров на борту
    \item \textbf{Pax\_fatalities} --- количество погибших пассажиров
    \item \textbf{Other\_fatalities} --- количество других погибших (на земле)
    \item \textbf{Total\_fatalities} --- общее количество погибших
\end{itemize}

\subsubsection{Техническая и эксплуатационная информация}
\begin{itemize}
    \item \textbf{MSN} --- Manufacturer Serial Number --- номер ВС от производителя
    \item \textbf{YOM} --- Year of Manufacture --- год производства
    \item \textbf{Location} --- местоположение крушения 
    \item \textbf{Country} --- страна крушения
    \item \textbf{Region} --- регион крушения (Европа, Азия и т.д.)
    \item \textbf{Flight\_number} --- номер рейса
\end{itemize}

\subsubsection{Данные о экипаже и воздушном судне}
\begin{itemize}
    \item \textbf{Captain\_Total\_flying\_hours} --- общий налёт капитана
    \item \textbf{Captain\_Total\_hours\_on\_type} --- налёт капитана на данном типе ВС
    \item \textbf{Copilot\_Total\_flying\_hours} --- общий налёт второго пилота
    \item \textbf{Copilot\_Total\_hours\_on\_type} --- налёт второго пилота на данном типе ВС
    \item \textbf{Aircraft\_flight\_hours} --- общий налёт воздушного судна
    \item \textbf{Aircraft\_flight\_cycles} --- количество циклов полёта ВС
\end{itemize}

\section{Результаты анализа}

\subsection{Анализ качества данных и пропущенных значений}

\begin{figure}[ht!]
\centering
\includegraphics[width=0.95\textwidth]{images/NaNCount.png}
\caption{Визуализация распределения пропущенных значений по всем переменным}
\label{fig:missing_values}
\end{figure}

На рисунке \ref{fig:missing_values} представлена комплексная визуализация пропущенных значений.
График показывает долю пропущенных данных для каждой переменной в наборе, что позволяет оценить
качество данных по каждому признаку.

К сожалению, наиболее интересные для анализа столбцы, связанные с информацией о налёте
капитана, второго пилота и воздушного судна, содержат слишком много пропусков, и от их
исследования придётся отказаться.

\subsection{Некоторая интересная статистика}

\subsubsection{Статистика по типам самолётов}

На рисунке \ref{fig:losses} изображена статистика по типам самолётов, наиболее часто
попадавших в катастрофы.

\begin{figure}[ht!]
\centering
\includegraphics[width=0.85\textwidth]{images/TopOfCrushedPlanes.png}
\caption{Самолёты, наиболее часто попадавшие в катастрофы}
\label{fig:losses}
\end{figure}

Первенство самолёта \verb|Douglas DC-3|, также известного под обозначением для
американских ВВС \verb|Douglas C-47 Skytrain|, связано с невероятной популярностью и
массовостью данной модели. За всё время выпуск составил более $16000$ единиц.
Это был основной военно-транспортный самолёт США во время Второй мировой войны. 

На втором месте также расположился самолёт времён Второй мировой войны: английский 
\verb|Bristol Blenheim|. Значительные потери этих самолётов могут быть связаны
с устареванием конструкции к началу войны и тяжестью боёв с более подготовленным
противником. 

Третьей по количеству крушений расположена польская модификация советского послевоенного
самолёта \verb|АН-2|, выпущенная серией около $12000$ машин (всего выпуск \verb|АН-2|
составил более $18000$ единиц). 

Аналогичная статистика, но по странам крушения, представлена на рисунке \ref{fig:countries_losses}.

\begin{figure}[ht!]
\centering
\includegraphics[width=0.85\textwidth]{images/TopOfCountriesCrushedPlanes.png}
\caption{Страны, в которых чаще всего происходили катастрофы}
\label{fig:countries_losses}
\end{figure}

В страны, в которых крушения происходили чаще, чем на нейтральной территории, входят
США, Великобритания, Россия (СССР), Франция, Германия и Канада. Самые развитые
и большие страны, в которых авиасообщение достаточно интенсивно в сочетании
с большой территорией. Кроме того, на территории Великобритании, России, Франции
и Германии шли интенсивные воздушные бои, в которых гибли многие самолёты.

Например, на рисунке \ref{fig:BoB_losses} изображена статистика потерь самолётов
в битве за Британию, проходившей с 10 июля по 30 октября 1940 года.

\begin{figure}[ht!]
\centering
\includegraphics[width=0.85\textwidth]{images/BoBCrushedPlanes.png}
\caption{Самолёты, погибшие в битве за Британию}
\label{fig:BoB_losses}
\end{figure}

Наибольшие потери несли английские \verb|Bristol Blenheim|, \verb|Handley Page HP.52|
\verb|Hampden| и немецкие \verb|Heinkel He.111|. При этом потери королевских ВВС превосходят
потери Люфтваффе, что не верно в общем контексте. Вероятно, в набор данных не попало большинство
немецких потерь, которые не были задокументированы. Кроме того, в исследуемом наборе не
отражены потери истребителей и ударных самолётов.

Если оценивать распределение потерь по годам, то выяснится, что самым интенсивным
с точки зрения авиакатастроф был $1942$ год (см. рис. \ref{fig:years_losses}). В это
время начинаются интенсивные налёты бомбардировщиков на территорию Германии,
ПВО которой ещё достаточно сильна для отражения атак. Этим и объясняются большие
потери в английских бомбардировщиках (см. рис. \ref{fig:1942_losses}), которые
занимают первые 9 строчек потерь.

На 10 же строчке расположились советские \verb|По-2|, легкомоторные медленные самолёты,
в основном использовавшиеся как почтовые. В целом можно отметить, что статистика
не включает информацию о многих советских самолётах.

\begin{figure}[ht!]
\centering
\includegraphics[width=0.95\textwidth]{images/YearsCrushedPlanes.png}
\caption{Потери самолётов по годам}
\label{fig:years_losses}
\end{figure}

\begin{figure}[ht!]
\centering
\includegraphics[width=0.95\textwidth]{images/1942CrushedPlanes.png}
\caption{Потери самолётов за 1942 год}
\label{fig:1942_losses}
\end{figure}

\subsubsection{Локации и фазы полёта, в которых чаще происходили крушения}

Имеет смысл разбить анализируемый набор данных на два: крушения до 1980 года и после.
С 1980 года заканчиваются основные военные конфликты и начинается переход к современной
системе управления полётами, что вносит значительный вклад.

На рисунках \ref{fig:bef80Loc} и \ref{fig:bef80Ph} можно видеть статистику
до 1980 года. Чаще всего крушения происходили во время полёта вблизи аэродромов
или на равнинной местности.

\begin{figure}[ht!]
\centering
\includegraphics[width=0.88\textwidth]{images/LocationCrusedPlanesBefore1980.png}
\caption{Место крушения самолёта}
\label{fig:bef80Loc}
\end{figure}

\begin{figure}[ht!]
\centering
\includegraphics[width=0.88\textwidth]{images/FlightPhaseCrusedPlanesBefore1980.png}
\caption{Фаза полёта, в которой произошло крушение}
\label{fig:bef80Ph}
\end{figure}

Статистика после 1980 года значительно отличается (см. рис. \ref{fig:aft80Loc} и \ref{fig:aft80Ph}).
Крушения всё также в основном происходят вблизи аэропортов, но уже значительно чаще, чем на равнинной
местности. Самой опасной фазой полёта становится посадка, что подтверждается известной
из других источников статистикой.  

\begin{figure}[ht!]
\centering
\includegraphics[width=0.95\textwidth]{images/LocationCrusedPlanesAfter1980.png}
\caption{Место крушения самолёта}
\label{fig:aft80Loc}
\end{figure}

\begin{figure}[ht!]
\centering
\includegraphics[width=0.95\textwidth]{images/FlightPhaseCrusedPlanesAfter1980.png}
\caption{Фаза полёта, в которой произошло крушение}
\label{fig:aft80Ph}
\end{figure}

\subsection{Статистические тесты}
В анализе были подготовлены следующие статистические тесты:
\begin{itemize}
    \item Проведён тест на нормальность распределения числа погибших;
    \item Построен доверительный интервал для математического ожидания числа погибших;
    \item Проверена гипотеза равенства математических ожиданий числа членов экипажа
    и погибших членов экипажа.
\end{itemize}

\subsubsection{Нормальность распределения числа погибших}

Q-Q кривая для числа погибших представлена на рисунке \ref{fig:QQ}. Данная кривая
значительно отклоняется от проведённой биссектрисы, поэтому можно утверждать,
что число погибших не распределено нормально.

\begin{figure}[ht!]
\centering
\includegraphics[width=0.7\textwidth]{images/QQTotalFatalities.png}
\caption{Q-Q кривая числа погибших}
\label{fig:QQ}
\end{figure}

Для дополнительной проверки был проведён тест Шапиро-Уилка, который
также показал, что распределение числа погибших не является нормальным.

\subsubsection{Доверительный интервал для числа погибших}

Построим доверительный интервал для математического ожидания с
доверительной вероятностью $0,95$. Для этого используется распределение
Стьюдента (т.к. дисперсия неизвестна). В итоге получим значения
$[4,75;5,05]$. 

\subsubsection{Равенство математических ожиданий числа членов экипажа и
числа погибших членов экипажа}

Для проверки равенства математических ожиданий двух различных распределений
используется двухвыборочный критерий Стьюдента. В итоге получим \verb|p-value|,
равное $0$, что значит, что математические ожидания не совпадают.

На рисунке \ref{fig:t2-test} представлено соотношение между жертвами среди
экипажа и членами экипажа на борту во время крушения.

\begin{figure}[ht!]
\centering
\includegraphics[width=0.7\textwidth]{images/CrewFatalities.png}
\caption{Число погибших членов экипажа и число членов экипажа на борту}
\label{fig:t2-test}
\end{figure}


\end{document}